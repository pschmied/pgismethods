\documentclass[12pt,letterpaper]{article} % for a short document

% Bibliography
\usepackage[authordate,strict,autolang=hyphen,bibencoding=inputenc,doi=true,isbn=false,annotation=false]{biblatex-chicago}
\addbibresource{../references/references.bib}


\title{Reproducible Methods for Urban Form Analysis: A Case Study of PostGIS}
\author{Phil Hurvitz, \ldots}
\date{}

%%% BEGIN DOCUMENT

\begin{document}

\maketitle

\section*{Background}
When delineating travel catchment areas or travel sheds, network-based
buffers are often preferable to circular, great arc buffers in that
network buffers more realistically reflect the distance limitations
imposed by different street patterns. \textcite{Forsyth2014sausage}
identified a threat to replicability for studies using such buffers,
due to changes over the years in how ESRI's market-leading GIS
software has calculated such service areas. In response,
\citeauthor{Forsyth2014sausage} proposed ``sausage'' network buffers,
created via a transparent algorithm, for use in measuring aspects of
urban form related to food and physical activity. Presented as
replicable and cross-platform, the implementation described in that
article nevertheless relies on ESRI's commercial GIS and Network
Analyst extensions, and is thus tied to the Microsoft Windows
platform. This paper validates the the sausage buffer technique with
particular attention paid to reproducibility and cross-platform
implementation using readily available free and open source
software. Then, because buffer creation is often an early step in a
larger analysis, we present a case study demonstrating the
complementarity of our sausage buffering implementation with a typical
follow-on analysis within the PostGIS platform.

% Let's mention something about reproducibility up front.
The methods presented here are in part a response to a growing call
from across numerous disciplines to ensure computer aided analyses are
reproducible. The basic standard for reproducibility is that the data
used are available, and that analyses are shared in the form of
computer code \cite{Peng2011computational}. In that sense, the methods
presented here are reproducible because analyses are executable code
rather than a set of manual procedures, and that those pieces of code
and the data are publicly available. Further, building these methods
atop free and open source software achieves a higher standard of
reproducibility by eliminating dependencies on any ``black box''
computer code. In principle, this means that every piece of software
used in these methods is open to scrutiny.

% Relevance to planning and epidemiology
Developing geoprocessing methods around free and open source tools
holds particular benefits for disciplines such as epidemiology, public
policy, and urban planning because of the reduced barriers of adoption
for practitioners and advocates outside of academia. Academic
institutions and larger public agencies may be able to afford software
licenses. However, smaller agencies or advocacy groups wishing to
perform, for example, walkshed analysis might otherwise be unable to
make use of analyses reliant on proprietary software.

% Not a literature review, per se. Since we are developing a methods
% paper largely in response to one particularly important method, I'm
% not sure how important a full-blown review is. Open to suggestions,
% however. This could potentially be split off and developed into
% something more complete. 
Finally, the PostGIS platform is especially beneficial in that it
naturally facilitates reproducible scripted analyses, as well as can
provide the core infrastructure for data management on a research
project. We are not the first to identify PostGIS for use in
research. In a keyword search for ``PostGIS'' on ISI Web of Science,
we identified 35 articles across a number of disciplines describing
their use of PostGIS. In the transportation sector,
\textcite{Wang2015routable} used PostGIS to store and analyze a large
number of GPS traces in order to infer characteristics about roadways
such as position and rules, for the purposes of generating routable
roadway networks. In another recent article,
\textcite{Brovelli2015FOSS} describe the use of PostGIS together with
GeoServer to collect data about roadway pavement conditions from the
general public. Notable in this study is the potential for PostGIS in
supporting digitally-mediated public participation.



\section*{Methods}
Using the process described by \textcite{Forsyth2014sausage}, we
implemented the sausage buffering technique using the PostgreSQL
relational database with PostGIS and PgRouting extensions. We then
implemented the same technique in ArcGIS Desktop X.XX, as a basis for
comparison with the results presented by
\citeauthor{Forsyth2014sausage}.

Using the two buffering implementations, we generated two sets of
sausage buffers around a common set of N points. These points were
randomly selected from non-water areas of <SOME GEOGRAPHY, probably
King County>. We then make several cross-comparisons between the two
sets of buffers, matching buffers according to a unique identifier
assigned to the initial set of points. First we compare the difference
in area between each set of buffers. Then, to assess the possibility
of spatial misalignment between similarly sized buffers, we computed
the symmetric difference of the two sets of sausage buffers and then
calculated the area of the non-overlapping regions.


\section*{Results}


\section*{Discussion}


\section*{Conclusions}


\printbibliography


\end{document}
